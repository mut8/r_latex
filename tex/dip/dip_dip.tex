\documentclass[a4paper,10pt]{book}
\usepackage[utf8x]{inputenc}
\usepackage{textcomp, fixltx2e}
\usepackage[authoryear, round, sort]{natbib}
\usepackage{setspace}
\usepackage{fullpage}
\usepackage{textgreek}
\usepackage{graphicx}
%\renewcommand{\thechapter}{}
\renewcommand{\chaptername}{}

\doublespacing
%\addtolength{\oddsidemargin}{-.5in}
%\addtolength{\evensidemargin}{-.5in}
%\addtolength{\textwidth}{1in}
%\addtolength{\topmargin}{-.5in}
%\addtolength{\textheight}{1in}

%opening
\title{High molecular weight compounds in beech litter decomposition}
\author{Lukas Kohl}

\begin{document}

\begin{titlepage}
\begin{flushright}
\includegraphics{RZ_Logo_Uni_sw.jpg}
\end{flushright}
\vspace{2cm}

\begin{center}
\Huge \textbf{DIPLOMARBEIT}\\
\vspace{1cm}
\normalsize Titel der Diplomarbeit\\
\large \textbf{``High molecular weight compounds \\ in beech litter decomposition''}\\
\vspace{1cm}
\normalsize Verfasser\\
\large \textbf{Lukas Kohl}\\
\vspace{1cm}
\normalsize angestrebter akademischer Grad\\
\large \textbf{Magister der Naturwissenschaften (Mag. rer. nat.)}\\
\end{center}
\vspace{5cm}

\begin{flushleft}
\normalsize Wien, 2011 \\
Studienkennzahl lt. Studienblatt: A444\\
Studienrichtung lt. Studienblatt: Ökologie\\
Betreuer: Ao. Univ.-Prof. Mag. Dr. Wolfgang Wanek\\
\end{flushleft}


\end{titlepage}

 
% Donna Haraway.
% \newpage
% 
% \tableofcontents
% 

% \chapter*{Acknowledgements}
% 
% My supervisors Andreas Richter and Wolfgang Wanek, who gave me the chance to change from greenhorn biologist to still kinda green /something/. It's amazing being given the time to gather knowledge and become specialist for a method.. I also have to thank them and especially Marianne Popp (Head of the department for chemical ecology and ecosystem research) for creating an environment that made me return to my old chemical obsession. 
% 
% Both them and all my coworkers at the department I thank for the amazingly cooperative mood at the department and the solidarity among diplma, phd students and post docs at the department. 
% 
% For vital support with Pyr-GC/MS I want to thank especially clemens Schwarzinger and Andreas Blöchl who taught me my fist steps with the method and Instrumentation, and Birgit Wild for both continous technical support and the right amount of sarcasm when needed. Especial thanks also go to Maria Mooshammer who knew all relevant literiture and every piece of data of our project by hard. 
% Further thanks go to Florian Hofhansl, and Tina Kaiser for vital advice on R programming and statistic questions when needed most. 
% Thanks to Jörg Schnecker, partner in crime on our new Pyroprobe instrument, for hundreds of small tricks for sample preparation. 
% 
% To the whole MicDiF team, who collected an enormous amount of data on our samples, providing me with all data needed and much more. Ieda Hämmerle-N. and Lucia Fuchslueger were nursing Mesocosm. Data applied in the work presented: Katherina Keiblinger (Stoichiometry, inorganic litter chemistry), Ieda Hämmerle-N. for enzyme data. ...
% 
% For financial support and understanding to my parents, and to the oeh bundesvertretung (2009-2011) for tolerating my absence at work. 
% 
% Finally to N.N., author of P.H.D. - Comics. I never would have made it without your work.

\chapter*{Introduction}


This work is composed of three major parts that serve different perposes and are aimed at different readers: 

\paragraph{General introduction} gives a broad introduction to litter decomposition and explains its importance in biogeochemical carbon and nutrient cycles. Plant polymers present in beech litter briefly presented. Furthermore, models and concepts used to describe litter decomposition and stoichiometrical approaches to describe decomposition processes are briefly reviewed. This chapter aims to give non-experts an compact overview over the background of this work. Finally, the experiment, of which this work forms part, is presented. 

\paragraph{Methodological comments} aims at future users of analytical pyrolysis shares experiences with this technique. It explaines the ratio behind choices taken in the implemention of our pyrolysis-GC/MS methods, points out alternatives, pitfalls encountered and possible further developments. It is not written not so much for professional analytical chemists as for biologists and other amateurs working with the method.

\paragraph{[name]}Is a manuscript written to be submitted to Soil Biology and Biochemistry, and aims to become a state-of-the-art original research article in a peer-reviewed international journal.

\paragraph{Appendix} presents additional resources and data, both as background of the research presented and as help to future pyrolysis applicators. Especially a complete peak list an several chromatograms that might be helpful for pyrogram peak assignment are presented.


\chapter{General Introduction}


\section{Litter decomposition and the global carbon cycle}

Rising atmospheric CO$_{2}$ concentrations and global climate changes caused by them \citep{IPCC2007pt1ch1} lead to an increased interest in natural carbon cycles and their anthropogenic modifications. Since pre-industrial times, annual means of athmospheric CO$_{2}$ concentration increased from 280 to 379 ppm (v/v) \citep{IPCC2007pt1ch1}. The source of approximately 80\% of the increase was be pined down to fossil fuel usage by comparing athmospheric CO$_{2}$ concentration to its \textsuperscript{13} C signature (fossil fuel C is depleted in \textsuperscript{13} C ) or corrensponding decrease in athmospheric O$_{2}$ concentrations\citep{IPCC2007pt1ch1}. Furthermore, land use change and cement production are accounted for additional CO$_{2}$ emissions. Between 2000 and 2005, mean annual CO$_{2}$ emissions from fossil fuel burning and cement production accounts for 7.2 $\pm$ 0.3 Gt CO$_{2}$-C. Additionally, land use change causes the annual emmissions 1.6 $\pm$ 1.1 Gt CO$_{2}$-C. Together with other greenhouse gases, elevated CO$_{2}$ concentrations are prognosted to raise earth mean surface temperature by nn degree by 20nn (IPCC?).

Anthropogenic CO$_{2}$ emissions are tightly interconnected with natural carbon cycles. Only 45\% of the emissions are found in the athmosphere, 30\% of the emmited CO$_{2}$ is absorbed in oceans and 25\% in terrestrial ecosystems . Oceanic CO$_{2}$ absorption is based on export of particular and dissolved organic carbon and dissolved inorganic carbon (HCO \textsubscript{3} \textsuperscript{-} , CO \textsubscript{3} \textsuperscript{2-} ) to intermediate and deep water layers. Land sinks take up carbon into larger vegetation- and soil C pools, i.e. due to a nothward shift of climatic limits for vegetation and CO$_{2}$ and N fertilization. However, a large part (-2.6 Gt a\textsuperscript{-1}) of this terrestrial carbon sinks is unaccounted for \citep[p. 515]{IPCC2007pt1ch7}. 

Finding this ``missing sink'' and prognosting feedback mechanisms of CO$_{2}$ emissions challenged scientists to strife for depthening their understanding of large scale biotic carbon transformation processes and storage. Globally, land plants assimilates 120 Gt C annually (gross primary production). This is almost one sixth of the global atmospheric CO$_{2}$ pool (750 Gt a-1) and more than 15 times more than antropogenic C emissions. Autotrophic (plant) respiration consumes one half of the assimilated carbon, the other half is introduced into decomposition process as plant litter. Animal biomass and herbivory form only a neglectable part of the total biomass [lit.], but can wield key controls on vegetation and its succession.

Ecosystem carbon balances are determined by the difference between carbon assimilation (photosynthesis) and respiration. While controls on photosynthesis rates are well understood, knowledge about decomposition processes is by far more limited. This is due to the fact that organisms capable of photosynthesis generally are green, seesile and grow aboveground, are therefore easy to find and study, while a large part of heterotrophic respiration is conducted by soil microbial communities of microscopic scale that dwell belowground, are  hard to identify, and live in a chemically complex environment. Due to the complex nature of soils, studying chemical transformation processes and chemical controls over microbial communities and physiology is easier in aquatic than in terrestrian habitates (for example, differences between nutrient contents and bioavailable nutrient amounts are smaller in aquatic environments, facilitating studies of nutrient control on microbial communities). However, research interest in terrestrial decomposition processes, especially litter decomposition, which sees the highest biomass turnover, is enormous, with more than one peer-reviewed research article per day published on litter decomposition between 2005 and 2009 \citep{Prescott2010}. 

Global litterfall summs up to for approximately 60 Gt C a\textsuperscript{-1}. Frequently between 30 and 70\% of this mass are lost in the first year and further 20 to 30 \% within another 5 to 10 years \citep[p.157]{Chapin2002}.

Temperate forests are highly productive, average net primary production is estimated for 1550 g m-2 a-1 (1/3 of which is allocated into belowground biomass). They cover 1.7 * 10\textsuperscript{7} km² (~ 1/15th of earth land surface) and account for 8.1 Gt a-1 NPP (1/8th of total terrestrial NPP) \citep[p?]{Chapin2002}. European beech (\emph{Fagus silvaticus} L.) is the dominant tree species in potential western and central europe. The distribution area is shown in figure nn. 

cultivation


Temperate forests 8.1 Gt a-1
Boreal forests 2.6 Gt a-1
Mediterranean shrublands 1.4 Gt a-1
Tropical savannas and grasslands 14.9
Temperate grasslands 5.6
Desserts 3.5
Arctic tundra 0.5 
Crops 4.1
sum 62.6 Gt

\section{Litter chemistry}
\subsection{Dramatis personae: Chemical constituents of beech litter}

\subsubsection{Carbohydrates}
Different carbohydrates are present in plant litter: 

\paragraph{Cellulose} 
is the \textbeta - 1-4 glycosidic polymer of glucose. It is no soluble under natural conditions. In beech litter, cellulose roughly accounts for roughly 20-25\% of litter dry mass \citep{Leitner2011}.

\paragraph{Hemicelluloses}
The term hemicellulose is used for a number of carbohydrate polymers present with cellulose in plant cell walls. These heterogenous compound class contains both pentoses and hexoses. 

Together, cellulose and hemicellulose form primary (plant) cell walls. 

\paragraph{Starch} 
is the \textalpha - 1-4 glycosidic polymer of glucose, with \textalpha 1-6 gylcosidic branches. In litter, only small amounts of starch are present (approximately 0.1\% dry weight in beech litter, \cite{Leitner2011}), but is among the most readily soluble compounds and therefore an important carbon and energy source especially in initial litter decomposition. 

\subsubsection{Lignin}

Lignin is a polymer composed of monomers derived from p-hydroxypropyl-phenol, with 0-2 methoxyl substitutions at C2 and C5. Different monomers are present in different plants: gymnosperm plants contain mainly guaiacol (1 methoxylic group) derived lignin, while angiosperm contains both guaiacol and syringol (2 methoxylic group) monomers. Grasses have high contents of not-methoxylated p-hydroxypropyl-phenol.

Unlike carbohydrate polymers, ligin polymerization is not based on enzymatic reactions, but (maybe after enzymatic activation) on radical reactions. Therefore, the polymerization process of lignin runs rather random, an activated lignin molecule can react with different other compounds present. Covalent interconnections between lignin and other cell wall compounds are common, especially with cellulose (then refered to as lignocellulose) and protein. Thereby, lignin can encapsulate these compounds, so that they are not available to microorganisms until the lignin cover is degraded.

Unlike carbohydrates and protein, lignin is not degraded by hydrolytic but by oxidative enzymes that produce hydroxylic radicals

\subsubsection{Tanin}
Tanins are ...

\subsubsection{Protein}

\subsubsection{Micronutrients}

\subsection{Cuticulary waxes}
Cuticulary waxes are composed of long chained aliphatic \textomega hydroxy-carboxylic acids, dicarboxylic acids and \textalpha - \textomega dihydroxyalkanes. These compounds form highly hydrophobic ester-bound polymers, which are recalcitrant due to their high hydrophobicity. 

\subsection{Analytical methods to characterize high molecular weight compounds}

\paragraph{Carbohydrates}
The characterization of sugar monomers present can be achieved by HPLC or GC analysis after hydrolysis (i.e. \cite{Snajdr2011}). Specific starch analysis is usually conducted by enzymatic hydrolysis with amylases, described by \cite{Leitner2011}. Cellulose and hemicelluloses are  often analyzed by selective extraction and gravimetry. This method is describe in more detail with lignin analysis.

\paragraph{Lignin}
Several methods have been developed to determine lignin, but [they are all problematic] \citep{Hatfield2005}.



\subsection{Substrate stoichiometry}

In ecology, the term ``stoichiometry'' is used to discribe the ratio, in which chemical elements are persent in an 




\section{Litter decompositions: Mechanisms and Models}
 Most simple models describe litter mass loss with a expenonential function:

\begin{equation}
L_T=L_0e^{kt}
\end{equation}

with \emph{L} standing for the Litter mass at time \emph{t} and $L_0$ for the initial litter mass. The constant t characterized the litter decomposition rate \citep[p.157]{Chapin2002} and allows comparison between different sites and litter. More complex models include different pools of litter biomass for chemical constituents of litter with different resistances to degradation. 

\subsection{Controls over decomposition rates}
\label{decomp_controls}

The three key controls on litter decomposition are climate, litter quality and decomposer community. \cite{Prescott2010} points out that these controls do not nessesarily indicate direct correlation between control and mass loss, but emphasises that for each controling factor there is (1) an optimum range, where decomposition is not limited by the factor, and decomposition rates are constrained to other controls, (2) a critical range, where decomposition is almost completely inhibited, and (3) an intermediate range between (1) and (2), where decomposition rates are correlated to the controling factor. 

Optimal climatic conditions for litter decomposition are between 60 and 75\% moisture at temperatures between 30 and 40 \textdegree C. Decomposition is substancially inhibited below 30\% and above 80\% moisture and below 10 \textdegree C temperature \citep{Prescott2010}. 

Litter quality is described by it's chemical constituents and nutrient contents. 

The three main processes in litter are (1) \emph{leaching} of soluble compounds from litter, (2) the \emph{fragmentation} of litter by by soil fauna and (3) the \emph{chemical alteration} of litter \citep[pp. 152f]{Chapin2002}. .

Litter decomposition rates are usually discribed with exponential equations, assigning a half-life time to either (1) bulk litter or (2) different half-life times to litter fractions. Traditionally, 3 carbon pools are 
\cite{Berg1980}

\section{Anthropogenic disturbances of litter decomposition}



% Litter decomposition controls the release of nutrients and the mineralization of assimilated carbon. 
% 
% Effects of increasiung atmospharic CO$_{2}$ on plants:
% 
% Increasing carboxylation (CO$_{2}$ assimilation) and decreasing oxigenation (O$_{2}$ assimilation). Both effects increase photosynthesis rates. (Farqzhar et al 1980, via IPCC AR4 2007 part 1 p. 195)
% 
% C3 plants generally show an increase in assimilation rates under elevated atmospheric CO$_{2}$ concentrations. C4 plants show no increase or an increase inferior to C3 plants, as they already posses a mechanism to concentrate CO$_{2}$ prior to RuBisCo assimilation. (IPCC AR4 2007 part 1 p. 195)
% 
% Higher atmospheric CO$_{2}$ concentrations improve plant water use efficiency (WUE). Longer growth seasons might be consequence.
% 
% \cite {Norby2001}
% 
% Better nitrogen use efficiency: less N needed in leaves to assimilation at the same rate. N is not nessesarily limiting under CO$_{2}$ fertilization conditions.
% Higher nitrogen fixation. 
% 
% Effects on decomposition processes:
% 
% It is 
% 
% %Ecosystem carbon balances are determined by the difference between assimilation (photosynthesis) and respiration. Organisms capable of photosynthesis are green, seesile and grow aboveground, and are therefore easy to find and study. In contrary, heterotrophic respiration is in a large part done by soil microbes that grow belowground, are of microscopic scale and hard to identify, and live in a chemically complex environment. Not surprisingly, knowledge of decomposition processes is far more incomplete than of assimilation. 
% 
% In aquatic ecosystems, especially in.. litter fall is a significant source of organic matter. 
% 
% Carbon assimilation by plants significantly influences global carbon cyclation. 
% 
% Global annual litterfall is estimated to size 60 Gt of organic carbon \citep[p. nn]{Chapin2002}. Approximately 90\% of litter carbon are respired, while 10\% are sequestered to soil organic matter (Prescott?). Compared to anthropogenic CO$_2$ fluxes, litter turnover involves approximately 10 times as much carbon as the fossile fuel combustion, while carbon sequestration to soils is of the same dimension.
% 
% Litter decomposition processes are likely to be affected by anthropogenous environmental changes, \cite{Couteaux1995} points out 4 influences: Elevated athmospheric CO$_2$ concentrations might cause a wider C:N ratio in litter, (2) higher temperature might lead to higher litter C mineralization rates (3) and to higher N mineralization rates (with consequences for the N availability to plants) and (4) north shift of vegetation might trigger carbon sources and sinks which are hard to predict.
% 
% \section{Litter incubation experiment - experimental design}
% 
% The current study focuses on the the chemical alteration of litter by the microorganisms. Therefore an experimental design was chosen that excludes soil fauna and aims to minimize leaching. We kept environmental conditions stable and in the optimal range as described above \ref{decomp_controls}
% 
% In the current study, we keep climate constant at 60\% moisture and and a temperature of 15 \textdegree C. We sterilize the litter and innoculate with a common innoculum to ensure an equal microbial community \emph{at the beginning} of the experiment. We trace how the microbial community develops out of this initial composition during the experiment.
% 



\chapter{Methodological comments}

\section{Column choice and temperature programm}
This work uses a Carbowax column (Supelcowax 10) for seperation. The column was chosen for better peak seperation after comparing several measuremts on this column with a RTX 35 (Restec) column. A good part of the published Pyrolysis-GC/MS studies use simple HP-5/SP-5 or similar standard GC columns.

However, during analysis, limitations of the column became evident, especially the limited temperature range (maximum temperature 280°C) of the column. Due to this and probably long retention of polar substances on the column, we were not able to detect several interesting compounds: long chained (C18+) n-alkyl-alcohols, \textomega - hydroxy - n-alkyl-fatty acids, and \textalpha - \textomega - n-alkyl-dicarboxylic acids (all common in cuticulary waxes). Our detection of n-alkanes and alkenes was limited to C27 compounds (C29 compounds could be detected, but strong discrimination against them was suspected). Among the carbohydrate products, only traces of leavoglycosan and no other dehydroxysugars. Laevoglycosan is usually among the major decomposion products of cellulose, and among anhydrosugars, products originating from different sugar monomeres can be differenciated, especially between pentoses and hexoses. 
Among the lignin products, pyrolysis products with functional groups in the side chain and syringol derivatives in general were discriminated against. 

The GC temperature program was designed to freeze - trap pyrolysis products at the beginning of the column. Therefore a low initial temperature was chosen (50 \textdegree C). The maximum temperature of the column according to the producer is 280 \textdegree C. Again, reaching a higher temperature would be of advantage, because larger molecules (which have high diagnostic value) would be detected.

\subsection{Internal standards and absolute quantification}

Quantifying pyrolysis products can be a challange itself: Beside the high number of complex products to be quantified, comercial availability of these substances is limited. Due to the low sample amounts (100-500 \textmu g) exact balancing of the sample is difficult, especially as pyrolysis vials are usually not optimized for balancing of to avoid sample losses. For the GSG Pyromat instrumentation, recovery rates strongly varied between samples, suposedly due to gas leakage in the Pyr-GC interface. Generally, reproductivity of recovery rates and balancing is not sufficiently high enough to relate absolute peak areas to sample inweight for quantitative analysis. 

Other chromatographic applications commonly exclude this ``injection bias'' by the use of an internal standard. Until now, this is not common in pyr-GC/MS analysis. Two recent publication add an internal standard to the sample: \cite{Steinbeiss2006} uses p-methoxyphenone, \cite{Bocchini1997} tests several substances and conclude that xx is most suited as an internal standard for lignin determination. In both approaches the internal standard is not chemically modified during pyrolysis but evaporated (``themal desorption'') and results in a single peak in the pyrogram. Internal standard amounts found can account for losses of pyrolysis products. It does not account for losses during the pyrolysis process itself, i.e. incomplete pyrolysis of the sample is not throughoutly heated to the intented temperature. Adding the internal standard to the sample in a known ratio is also difficult: usually the internal standard is applied by pipetting a small amount of a solution onto the sample (1-5 \textmu L). Larger volumes do no fit into the pyrolysis vials and often provoce leakage of the solution from the bottom-open vials.

A substancial part of the products formed by the pyrolysis of natural organic polymers are not or not exactly identified, commercial availability of pyroylsis products is limited. Also, if their thermal stability is insufficient, these substances can not be induced to the chromatic system by thermal desorption in the pyrolysis unit. Due to this problems and the high number of compounds produced, no publication quantifying single pyrolysis was published yet.

Quantifying substances of origin of pyrolysis products is even harder than quantifying the products themselves. For plant material, the most 1important classes of compounds analyzed - carbohydrates and lignin - are present in different forms in plant litter. However, especially for Carbohydrates can not be distinguished by pyr-GC/MS, but it has to be assumed that during pyrolysis they do not produce the same product in the same ratios. Lignin components different among plant families, reference material for angiosperm is scare. Chemical alternations in lignin structures are unavoidable during preparaation.

%discribe referencing Lig/CH

Due to the reasons above, commonly, analytical pyrolysis studies do not aim for an absolute quantification of pyrolysis products or their substances of origin. 



\subsection{Peak assignment}

Peak assignment is the crucial step in the analysis of pyr-GC/MS data. Usually not the whole dataset, but a small number of representative files are screened. 

For the current litter analysis, one replicate of initial litter and litter after 15 month incubation (from two different litter types) were analyzed. However, it was known from previous studies that litter types were highly similar in their composition. For more heterogenous samples, at least one replicate for each treatment should be analyzed. 

The following steps were applied:

A List all peaks over a certain area treshhold was compiled. This is done by (1) automatic integration with the Xcalibur Qual Browser and (2) manual screening of print-outs of the chromtogram. Initial air contamination peaks are excluded. These are usually between 0.8 and and 2 minutes GC runtime, have characteristic molecule (M+ ) ions at m/z 28 (N2) 32 (O2) and 44 (CO2) and are often by far the highest peaks in the pyrogram. 

An attempt to identify peaks with a relative peak area over a critical treshhold (i.e. 0.1 \% total peak area).

When one substance class is detected, missing pyrolysis products from the same substance of origin are looked for, usually using their most abundant MS fragments.

Finally, critical diagnostic peaks can be found when looked for (specific ion traces)


For plant material, \cite{Ralph1991} presents the most relevant data for the identification of pyrolysis products. The confirm the identity of over 100 pyrolysis products by standard addition. 

\subsection{Peak classification}



\paragraph{Lignin}


% \chapter{My fancy title}\footnote{whenever this gets ready, this paper will be submitted to...}
% % 
% % % Footnote: This paper is submitted to...
% % 
% % 
% \section{Abstract}
% 
% \section{Introduction}
% \input{introduction.tex}
% \input{materials.tex}
% \input{results.tex}
% 
% \section{Discussion}

\subsection{Intra-specific variance in beech litter and decomposition trends}

We find characteristic patterns of pyrolysis products from different sites. Most important differences were found between furane-type and cyclopentenone-type carbohydrate markers. Also, among the lignin markers, we found differences in the methylguaiacol:guaiacol and methylsyringol:syringol ratio. Differences in the carbohydrate pools possibly origin in different carbohydrates present in litter, while differences in lignin markers maybe indicate different polymerization structures. Alternatively, they can be result of matrix effects during pyrolysis. 

These differences were preserved during litter decomposition, probably due to the low litter decomposition speed observed in beech litter.
%sind eigentlich results

\subsection{Nutrient controls on carbon chemistry}

We found profound differences in patterns of accumulation and depletion of lignin and carbohydrates. During the first 6 month of decomposition, lignin is accumulated and carbohydrates are deplete in three litter types (KL, OS, SW). However, no litter carbon mineralization was not coupled to lignin accumulation or carbohydrate depletion in the forth litter type (AK). Comparing changes in litter chemistry to respiration rates (fig. \ref{fig:lci} (B) and \ref{fig:timeseries} (right side)), we can exclude low litter turnover as a reason for the missing shift in litter chemistry in AK . Especially as OS hat only slightly higher accumulated respiration, but a similar rate of lignin accumulation like SW and KL. This indicates, that - in contrast to the other litter types - there is no microbial substrate preference of carbohydrates over lignin in AK litter and that lignin is decomposed during early litter decay in AK. In other sites, lignin is not decomposed or only at a rate relatively slower than carbohydrates. 

Potential enzyme activities support our findings: N-rich sites had the highest absolute activity for both cellulase and oxidative enzymes. This reflects higher turnover of organic carbon in N-rich litter observed in most decomposition processes [lit maria?]. Unlike some other studies (reviewed by \cite{Sinsabaugh2010} - [check if fertilization experiments]) we did not find an inhibition of oxidative enzymes in absolute terms under high (natural) N content in the substrate. The absolute amount of enzymes produced [might be] limited by N availability and is strongly correlated with other decomposition processes [provide stats]. Unlike the absolute amount of enzymes produced, the ratio between cellulose hydrolyzing and oxidative enzymes is lower in AK than in other sites. Investments of the microbial community are directed more into degrading lignin in AK than in other sites.

%Unlike cellulose and protein, degradation of lignin does not yield a single specific monomer. Due to this unspecific biochemistry, it is not possible to specifically measure lignin decomposition speed by a pool dilution method. Nevertheless, the ratio between glucose depolymerization and respiration allows an estimation, to which extent non-glucose carbon is respired by litter microbes. 

%Several independend methods show similar indication: analytical pyrolysis, calculation of non-glucose respiration, potential enzyme activities. 

The early lignin decomposition concept recently presented by \cite{Klotzbucher2011} seems fit for one litter type (AK), but not for the other three. Several possible reasons for stimulated/inhibited lignin decomposition were suggested in recent literature:

(1) Litter nitrogen content was strictly correlated to most decomposition processes measured [enzymes, N-depoly, Glucose-depoly, ... ] after 6 month and [test!]correlated to respiration at earlier harvest. Earlier analysis of decomposition processes in the same samples found controls of N content and litter C:N ratios over decomposition processes \citep{Mooshammer2011, Leitner2011}. [The system is N limited, at least after 6 month.] However, N content is similar in AK and OS, so N content as a single factor can not explain the differences observed.

(2) The same applies for litter DOC content: Higher DOC quantity in SW and AK lead to different trends, in SW lignin was most accumulated in AK the least.

(3) Micro-nutrients are nessesary cofactors for oxidative enzymes and have different contents in the four litter types. Their availability can limit lignin degradation [lit]. However, in AK, their concentration in lower  (Mn, Fe) or equal (Zn) concentrations than in other litter types. Low contents of these Elements would explain inhibited, not enhance lignin decomposition in AK.

We therefore suggest that the ratio between microbially accessible (=dissolved) carbon and litter nitrogen content 

%other lipophilic compounds

\subsection{Changes in decomposition controls over time}

While we found no explaining factor for the initial amount of extractable carbon [beside a loose correlation to litter N content], DOM production is strictly correlated to nitrogen content after six month incubation. Initial DOM amounts show a high independence from other factors [including starch content [check]], DOM production or consumption surpluses increase or decrease the DOM pool during the first 6 month of incubation but then reach an equilibrium point at which DOM content correlates with litter N content.

Nitrogen content is also tightly correlated to respiration beyond 6 month incubation. Unlike proposed by \cite{Klotzbucher2011}, in our experiment respiration was not to be principally controlled by DOC i.e. labile carbon availability, but either both processes are controlled by nitrogen availability or respiration depends on available carbon, which itself is controlled by nitrogen availability as described above. Direct N limitation seems plausible, as de-polymerization of POM compounds depends on extracellular enzymes. Their produce requires large investments of nitrogen from the microbial community. 


Long chain alcanes are among the substance with had the highest increase during the first month of litter decomposition. During the first 3 month their relative peak area increased by 80\%. [Where does these compounds come from?] Fatty acids were the most important inpurity of isolated lignin fractions. They were decomposed faster than lignin, with little differences between litter types [faster in N-poor litter]

%The ratio between glucan depolymerization and respiration shifts between 97 and 181 days. while during the first two harvests, respiration is (relatively) higher in litter types with high DOC and low N content, after 181 days, this ratio is strictly correlated to litter N content, with higher respiration for sites with high N. This ratio allows different interpretations: It might indicate a higher carbon use efficiency on part of microbial communities with a low depolymerization:respiration ratio, or the use of alternative (non-glucose substrates) in litter types with a high ratio.


%H2 - H3. glc depoly (+ aa depoly?) : resp. 

%Another possible explaination of differences between proximate analysis and specific determination of lignin oxidation or pyrolysis products is that first steps of lignin degradation remove characteristic groups from lignin polymers (i.e. methoxy groups), leaving a rest lignin with no specific tracers recognizable with the methods mentioned. This would lead to an underestimation of lignin.

\subsection{Microbial biomass [and decomposition processes]}

After 6 month, AK shows the strongest increase in microbial C. The increase in microbial N is even stronger,  so that after 6 month, AK, a litter type with low N content has the highest microbial N content and the most narrow microbial C:N ratio. This is the time point, when the most lignin is decomposed in AK. We suggest, that this is due to better nitrogen accessibility after increased lignin decomposition. Dissolved C and N pools (organic and inorganic) are one magnitude smaller than microbial biomass pools, and can not harbor de-polymerized litter biomass, which must be (a) respires, (b) incorporated into biomass or (c) immobilized to the POM pool. 

Between 6 and 15 month, lignin does not further accumulate in any site. Microbial metabolisms are adjusted to their substrate, DOC production and consumption are in equilibrium.

Decomposition processes are well correlated to each other and litter N content.We did not, however, find feedback from elevated/depleted lignin content of processes measured.

CN ratios are consistent with the proteomic Fungi/Bacteria ratio. 
% \section{Conclusions}
% 
%\conclusions
%% \conclusions[modified heading if necessary]
\cite{Fontaine2011} suggests a ``bank model'' for SOM vs. litter degradation in soils. N-rich recalcitrant carbon is decomposed when N content is low, while N-fertilized soils principally degrade carbohydrate-rich and N-poor litter leachates. This leads to increased N mobilization in N-poor soils when additional (labile) carbon is available. On term of the microbial community, the production of oxidative enzymes is needed to degrade SOM, so investment in the production of these enzymes is up-regulated under C-rich and N-poor conditions. Our results suggest, that similar controls exist in litter decomposition.

%[?Lignin is not rejected for its intrinsic recalcitrance, but because it has little to offer to a community with rare labile carbon..?...]

\section{Acknowledgements}
FWF NRN MicDiF, Katherina Keiblinger DOC fForte
Thanks to Andreas Bl\"ochl, Clemens Schwarzinger, and Birgit Wild for technical advice on Pyr-GC/MS techniques.

\bibliographystyle{plainnat}
\bibliography{library}

% \input{graphs_captions.tex}
% \newpage
\begin{table*}[p] 
\begin{tabular}{ccccccc}
Litter type & LCI increase & initial DOC & N & P & Mn & Fe \\
\hline
AK&-&++&-\-&-&-\-&- -\\
KL&++&-\-&-&-\-&+&- -\\
OS&++&-\-&- -&+&-&++\\
SW&++&++&++&++&++&-\-\\
\hline
\end{tabular}
\caption{Summary of context data}
\label{tab:summary}
\end{table*}

% \input{tables.tex}

\end{document}

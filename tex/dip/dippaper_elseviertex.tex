
%% This is file `elsarticle-template-2-harv.tex',
%%
%% Copyright 2009 Elsevier Ltd
%%
%% This file is part of the 'Elsarticle Bundle'.
%% ---------------------------------------------
%%
%% It may be distributed under the conditions of the LaTeX Project Public
%% License, either version 1.2 of this license or (at your option) any
%% later version.  The latest version of this license is in
%%    http://www.latex-project.org/lppl.txt
%% and version 1.2 or later is part of all distributions of LaTeX
%% version 1999/12/01 or later.
%%
%% The list of all files belonging to the 'Elsarticle Bundle' is
%% given in the file `manifest.txt'.
%%
%% Template article for Elsevier's document class `elsarticle'
%% with harvard style bibliographic references
%%
%% $Id: elsarticle-template-2-harv.tex 155 2009-10-08 05:35:05Z rishi $
%% $URL: http://lenova.river-valley.com/svn/elsbst/trunk/elsarticle-template-2-harv.tex $
%%
%%\documentclass[preprint,authoryear,12pt]{elsarticle}

%% Use the option review to obtain double line spacing
\documentclass[authoryear,preprint,review,12pt]{elsarticle}

%% Use the options 1p,twocolumn; 3p; 3p,twocolumn; 5p; or 5p,twocolumn
%% for a journal layout:
%% \documentclass[final,authoryear,1p,times]{elsarticle}
%% \documentclass[final,authoryear,1p,times,twocolumn]{elsarticle}
%% \documentclass[final,authoryear,3p,times]{elsarticle}
%% \documentclass[final,authoryear,3p,times,twocolumn]{elsarticle}
%%\documentclass[final,authoryear,5p,times]{elsarticle}
%\documentclass[final,authoryear,5p,times,twocolumn]{elsarticle}
\usepackage{textcomp, fixltx2e}

%% if you use PostScript figures in your article
%% use the graphics package for simple commands
%% \usepackage{graphics}
%% or use the graphicx package for more complicated commands
%% \usepackage{graphicx}
%% or use the epsfig package if you prefer to use the old commands
%% \usepackage{epsfig}

\usepackage{fullpage}

%% The amssymb package provides various useful mathematical symbols
\usepackage{amssymb}


%% The amsthm package provides extended theorem environments
%% \usepackage{amsthm}

%% The lineno packages adds line numbers. Start line numbering with
%% \begin{linenumbers}, end it with \end{linenumbers}. Or switch it on
%% for the whole article with \linenumbers after \end{frontmatter}.
%% \usepackage{lineno}

%% natbib.sty is loaded by default. However, natbib options can be
%% provided with \biboptions{...} comman. Following options are
%% valid:

%%   round  -  round parentheses are used (default)
%%   square -  square brackets are used   [option]
%%   curly  -  curly braces are used      {option}
%%   angle  -  angle brackets are used    <option>
%%   semicolon  -  multiple citations separated by semi-colon (default)
%%   colon  - same as semicolon, an earlier confusion
%%   comma  -  separated by comma
%%   authoryear - selects author-year citations (default)
%%   numbers-  selects numerical citations
%%   super  -  numerical citations as superscripts
%%   sort   -  sorts multiple citations according to order in ref. list
%%   sort&compress   -  like sort, but also compresses numerical citations
%%   compress - compresses without sorting
%%   longnamesfirst  -  makes first citation full author list
%%
%% \biboptions{longnamesfirst,comma}

 \biboptions{round}

 \journal{Soil Biology and Biochemistry}

\begin{document}

\begin{frontmatter}

%% Title, authors and addresses

%% use the tnoteref command within \title for footnotes;
%% use the tnotetext command for the associated footnote;
%% use the fnref command within \author or \address for footnotes;
%% use the fntext command for the associated footnote;
%% use the corref command within \author for corresponding author footnotes;
%% use the cortext command for the associated footnote;
%% use the ead command for the email address,
%% and the form \ead[url] for the home page:
%%
%% \title{Title\tnoteref{label1}}
%% \tnotetext[label1]{}
%% \author{Name\corref{cor1}\fnref{label2}}
%% \ead{email address}
%% \ead[url]{home page}
%% \fntext[label2]{}
%% \cortext[cor1]{}
%% \address{Address\fnref{label3}}
%% \fntext[label3]{}

\title{Resource control over early lignin decomposition in beech litter}

%% use optional labels to link authors explicitly to addresses:
%% \author[label1,label2]{<author name>}
%% \address[label1]{<address>}
%% \address[label2]{<address>}

%\author{}

% \author[1]{Lukas Kohl}
% \author[1]{Maria Mooshammer}
% \author[1]{Sonja Leitner}
% \author[1]{Ieda H\"ammerle-N.}
% \author[1]{Alexander Frank}
% \author[1]{Lucia Fuchslueger}
% \author[1]{J\"org Schnecker}
% \author[2]{Katharina Keiblinger}
% \author[2]{Sophie Zechmeister-Boltenstern}
% \author[1]{Wolfgang Wanek}
% \author[1]{Andreas Richter}
%%%>>>>corresponding author?!
\author[1]{me}
\author[2]{and my friends}

\address[1]{Checo}
% \address[2]{BFW}
\address[2]{not Checo}


\address{}


\begin{abstract}
The degradation of plant polymers in litter decay determines quality and quantity of recalcitrant carbon input to soils. Nevertheless, the involved transformations remain unclear. Beech litter from four different sites, varying in N, P and inital soluble C content was incubated over 15 month. We follow the accumulation or depletion of lignin, carbohydrates and other lipophilic compounds by pyrolysis-GC/MS. 
In three of the four sites, we found a clear preference of carbohydrates over lignin decomposition during the first 6 month. The fourth site, characterized by low N and high soluble C content, shows no such preference, degrading lignin at the same rate as bulk litter biomass.


\end{abstract}

\begin{keyword}
%% keywords here, in the form: 
litter decomposition \sep lignin \sep analytical pyrolysis \sep Py-GC/MS

%% MSC codes here, in the form: \MSC code \sep code
%% or \MSC[2008] code \sep code (2000 is the default)

\end{keyword}

\end{frontmatter}

%\linenumbers

%% main text
\section{Introduction}

Plant litter biomass is dominated by macromolecular compounds. Together, lignin, carbohydrate and protein polymers make up xx\% of litter dry mass, while leach-able substances (``DOM'') in litter account for only xx \%. The conversion of insoluble compounds in particular organic matter (``POM'') into soluble substances is key process in litter decomposition: Microorganisms can only metabolize DOM directly, but rely on the excretion of extracellular enzymes to convert POM in DOM \citep{Klotzbucher2011,Bengtson2007,Marschner2003}.

Carbohydrates and protein polymerization is exactly controlled by catalytic enzymes. Lignin is result of a radical polymerization reaction of enzyme-activated hydroypropylphenol monomers. During lignin condensation other compounds - most prominently carbohydrates and protein - are incorporated into lignin structures \citep{Achyuthan2010}. Extracted (ADF) lignin fractions from fresh beech litter were found to have nitrogen contents twice as high as in bulk litter \citep{Dyckmans2002}. 

Conventional litter decomposition models [lit] follow the idea that macromolecules in litter form three independent carbon pools of increasing recalcitrance. These pools are attributed to (1) soluble compounds (most prominently starch), (2) cellulose and hemi-celluloses and (3) lignin. During decomposition, soluble compounds are easiest accessible for microbes and consumed first, followed by carbohydrates (i.e. cellulose). Lignin can be decomposed only by specialists and is not degraded until accumulated to a certain, critical level when it inhibits the degradation of other compounds \citep{Berg1980, Couteaux1995, Moorhead2006}.[more lit.] 

%[This concept was first described by , and stills forms the base of recent models and textbooks [lit!!hh]. 

One reason for the popularity of this model is that sizes of the three carbon pools can easily determined by proximate analysis. In these methods, litter cellulose, hemi-celluloses and lignin content are determined by sequential extractions with selective solvents. Especially for lignin determination, these methods (``Klason''- and ``ADF''-lignin) were repeatedly criticize as unspecific \citep{Hatfield2005}. When analyzed with alternative methods (NMR, CuO-oxidation, Pyrolysis-GC/MS), extracted lignin fractions contain many other than the proclaimed substances. (i.e. \cite{Preston1997}, [lit CuO], lit[Pyr]). However, while not helpful in tracking the fate of lignin in litter decomposition, these fractions - re-labeled ``acid un-hydrolyzable residues'' (AUR) - can be used as an indicator for the content of the most recalcitrant carbon compounds in litter \citep{Prescott2010}. For the lipids and plant waxes also found in AUR fraction, neither their accumulation/depletion during decomposition nor the effect of their concentration on decomposition processes is known [!check!,!lit!]

%[The question of reference material]

%While it is possible to determine carbohydrate and protein contents by enzymatic or chemical hydrolysis and quantification of monomers, due to the arbitrary structure of it's polymers, this approach can not be used for lignin determination.

%Especially lignin determination by this method has been citized extensively \citep{Hatfield2005}, as acid unsoluble residues (AUR, the fraction formally known as lignin) contain a number of hydrophobic compounds like cutin, surface waxes [fatty acids] and condesed tannins (lit). 

Recent studies using more specific methods to determine litter lignin content (CuO - oxidation, pyr-GC/MS, NMR) question the previously assumed intrinsic recalcitrance of lignin. Mean residence times for lignin in soils were calculated from both laboratory and outdoor incubation of litter/soil mixtures. Lignin residence times found were no longer than other carbon compounds or bulk SOM \citep{Thevenot2010a, Bol2009} [more lit?]. For litter, lignin decomposition rates were found not to increase from early to late decomposition stages \citep{Klotzbucher2011}. Based on these results, the authors propose a new model for lignin degradation: fastest lignin degradation in litter decay occurs during early litter decomposition; lignin decomposition during late decomposition is limited by (dissolved organic) carbon availability. 

Neither the traditional 3-pool models nor \cite{Klotzbucher2011} elaborate the effect of macro-nutrient (nitrogen and phosphorous) availability on lignin decomposition. Nitrogen fertilization experiments on litter and soils suggest that litter nitrogen content affects lignin degradation: N addition increases mass loss rates in low-lignin litter while slowing down decomposition in lignin-rich litter \citep{Knorr2005}. High nitrogen levels were reported to inhibit lignolytic enzyme in forest soils\citep{Sinsabaugh2010}. Cellulose addition lead to a higher mineralization of SOM in fertilized than in unfertilized soils \citep{Fontaine2011}. However, results of artificial fertilization can not be compared to different ``natural'' nutrient gradients. To our knowledge, no other experiment has yet compared effects of intra-specific variance in litter nutrient contents on decomposition processes. N-fertilization experiments can simulate increased N-deposition rates. To simulate variations litter C:N ratios, our approach is preferable, because potential changes in litter N content will most probably affect complex POM substrates. There, N location and accessibility is different of the low molecular weight N species available for fertilization experiments. 


% [Elevated N deposition and elevated soil N content increase litter N contents, while a recent meta-study hints that elevated atmospheric CO$_{2}$ concentrations cause wider litter C:N ratios \citep{Luo2006}. Therefore it is important to assess the impact of shifts in litter C:N ratio on decomposition processes and the chemical nature of the resulting organic matter to predict feedback mechanisms of anthropogenic alterations of global carbon and nitrogen cycles. While predictions of changes in mass loss rates under alternated litter C:N ratios are abound [lit.], no studies on changes in the quality of litter biomass during and after decomposition or of the dynamics of accumulation/depletion of fragments of the litter biomass during decomposition exist yet.]

%Also, interdependance of these pools seems much higher than expected. Lignified cellulose and protein is not accessable without degrading lignin... Ligin fractions extracted from (fresh) beech litter have beed demonstrated to have a very narrow C:N ratio if compared to bulk litter, nonligneous cell walls or soluble matter. It is assumed, that proteins are covalently bound to lignin polymers during polymerization. For litter decomposing microbia this implies that degrading lignin increases nitrogen availability, while degrading non-lignified carbohydrates yields more metabolic energy, but no additional nitrogen. In an incubation eperiment with a litter/soil mixture, N addition stimulated carbon mineralization during the first weeks of incubation but slowed it down duringThe after  with  ratios after climate chamber incubation for up to 15 month. late decomposition.

%In temperate forests, litter decomposition is generally considered nitrogen limited (lit).
%several experiments report retention times for lignin in soil [not much higher than other soil components].

Several recent studies apply analytical pyrolysis (Pyr-GC/MS) to characterize complex natural organic polymers like soil organic matter (SOM, \cite{Vancampenhout2010}[more lit here]). 
%Only a limited number of pyrolysis studies comparing different decomposition levels have been done. 
Pyrolysis-based decomposition studies usually focus on the woody material \citep{Vinciguerra2007} or  soil/litter mixtures [lit! - z.b. gleixner ca.1999]. Microbial decomposition of straw was followed by Pyr-GC/MS by [lit] We found only one study analyzing different stages of litter decomposition with analytical pyrolysis \citep{Franchini2002} and one recent study using a related technique (thermally assisted hydrolysis and methylation in \citep{Snajdr2011}. 

\cite{Kuder1998}

%European beech (\emph{Fagus silvaticus} L.) is the dominant forest building tree species in central europe. 
In this study we analyze samples of climate-chamber incubated beech litter varying in N and P content with Pyrolysis-GC/MS (pyr-GC/MS). The experiment was designed to study microbial decomposition, exclude decomposing fauna and keep climatic conditions constant. Extensive data on litter chemistry and and decomposition process rates are available for this samples from previously publications \citep{Mooshammer2011, Wanek2011, Leitner2011} as unpublished data [provided by the MicDiF national research network]. 

We focus on changes in lignin and carbohydrate content, assuming that

(1) Lignified biomass and non-lignified carbohydrates are alternatively degraded. Microorganisms have to allocate N in the production of different enzymes. When environmental conditions are constant, microbial substrate preference is determined by litter chemistry,

(2) While (non-lignified) carbohydrates are easier degraded than lignin and the resulting sugar monomers yield more energy, lignin degradation improves to accessibility of nitrogen (``lignin mining'', \cite{Craine2007}). 
 More lignin is decomposed when nitrogen availability is low, and high nitrogen availability inhibits lignin degradation.

(3) Lignin degradation is inhibited when little DOC is available and decomposition is energy limited (as proposed by \cite{Klotzbucher2011}). 

%Their accumulation and/or depletion during decomposition was extensively studied, nevertheless recent studies based on new methods to measure lignin content challange traditional models.

% Decomposition conditions control whether assimilated carbon is released immideately or sequestered to soil organic matter. Decomposition processes also control the quality of the resulting organic matter and therefore soil carbon recalcitrance. 

%\cite{Prescott2010} suggests the amount of remaining biomass more important than the decomposition speed. We suggest, the chemical nature of this remaining biomass might be a factor too. we should study the influences of environmental parametres on the chemical conposition of the remaining biomass. 

% but using methods based on detergent extraction no regarded unspecific. 

%Large scale experiments (lit!) conducted during the last years collected detailed data on chemical and ecological controls predict dry mass loss during litter decomposition. Nevertheless, understanding of chamical transformations in litter loss is still limited, as most studies limit their analysis of high molecular weight compounds to detergent extraction based methods (Klason Lignin, ... - lit). 


%\cite{Snajdr2011a} used THM (a pyrolysis-related method) in a litterbag expiriment, but limits analysis to determining lignin:carbohydrate ratios. 

%These studies often explain biomarkers found in SOM by their occurance in a plant polymers and propose a decendence from thos biomarkers \citep{Schellekens2009b}, some of them also analyze local vegetation for comparison.


% We follow three research questions:

%(1) To which does HMW carbon at different sites differ? 

%(2) Which substance classes are accumulaten/depleted during decomposition and is the rate of accumulation/depletion influences by nutrient compositoin? Nutrient addition studies show a decrease of phenole oxidizing enzymes (lit!), but those have not been shown in pyrolysis data yet.

%(3) Does HMW chemistry excercice control over decomposition processes? Do changes in HMW chemistry (i.e. accumulation of more recalcitrant substances) explain the decline of decomposition rates?




% Nitrogen availability has been shown to be rate limiting in beech litter (lit) ??\citep{Mooshammer2011}??.

%With increasing anthropogenic nitrogen deposition and contradicting studies on whether litter C:N ratio is change due to increasing anthmospheric CO$_2$ ratios, knowledge about effects of nutrient availability on the chemical properties of soil organic matter become increasingly relevant (lit.).

% Most experiments study nutrient control by fertilization. 
%Nutrient availability Litter stoichiometry...
%prescott -> metastudy zu litter CN aus den 90ern
%die aktuelle metastudy
%prescott N deposition 
%N-addition studys. few studies on stoichiometry in naturally variing systems.

% CN ratio changes >> 


\input{materials.tex}
\input{results.tex}

\section{Discussion}

\subsection{Intra-specific variance in beech litter and decomposition trends}

We find characteristic patterns of pyrolysis products from different sites. Most important differences were found between furane-type and cyclopentenone-type carbohydrate markers. Also, among the lignin markers, we found differences in the methylguaiacol:guaiacol and methylsyringol:syringol ratio. Differences in the carbohydrate pools possibly origin in different carbohydrates present in litter, while differences in lignin markers maybe indicate different polymerization structures. Alternatively, they can be result of matrix effects during pyrolysis. 

These differences were preserved during litter decomposition, probably due to the low litter decomposition speed observed in beech litter.
%sind eigentlich results

\subsection{Nutrient controls on carbon chemistry}

We found profound differences in patterns of accumulation and depletion of lignin and carbohydrates. During the first 6 month of decomposition, lignin is accumulated and carbohydrates are deplete in three litter types (KL, OS, SW). However, no litter carbon mineralization was not coupled to lignin accumulation or carbohydrate depletion in the forth litter type (AK). Comparing changes in litter chemistry to respiration rates (fig. \ref{fig:lci} (B) and \ref{fig:timeseries} (right side)), we can exclude low litter turnover as a reason for the missing shift in litter chemistry in AK . Especially as OS hat only slightly higher accumulated respiration, but a similar rate of lignin accumulation like SW and KL. This indicates, that - in contrast to the other litter types - there is no microbial substrate preference of carbohydrates over lignin in AK litter and that lignin is decomposed during early litter decay in AK. In other sites, lignin is not decomposed or only at a rate relatively slower than carbohydrates. 

Potential enzyme activities support our findings: N-rich sites had the highest absolute activity for both cellulase and oxidative enzymes. This reflects higher turnover of organic carbon in N-rich litter observed in most decomposition processes [lit maria?]. Unlike some other studies (reviewed by \cite{Sinsabaugh2010} - [check if fertilization experiments]) we did not find an inhibition of oxidative enzymes in absolute terms under high (natural) N content in the substrate. The absolute amount of enzymes produced [might be] limited by N availability and is strongly correlated with other decomposition processes [provide stats]. Unlike the absolute amount of enzymes produced, the ratio between cellulose hydrolyzing and oxidative enzymes is lower in AK than in other sites. Investments of the microbial community are directed more into degrading lignin in AK than in other sites.

%Unlike cellulose and protein, degradation of lignin does not yield a single specific monomer. Due to this unspecific biochemistry, it is not possible to specifically measure lignin decomposition speed by a pool dilution method. Nevertheless, the ratio between glucose depolymerization and respiration allows an estimation, to which extent non-glucose carbon is respired by litter microbes. 

%Several independend methods show similar indication: analytical pyrolysis, calculation of non-glucose respiration, potential enzyme activities. 

The early lignin decomposition concept recently presented by \cite{Klotzbucher2011} seems fit for one litter type (AK), but not for the other three. Several possible reasons for stimulated/inhibited lignin decomposition were suggested in recent literature:

(1) Litter nitrogen content was strictly correlated to most decomposition processes measured [enzymes, N-depoly, Glucose-depoly, ... ] after 6 month and [test!]correlated to respiration at earlier harvest. Earlier analysis of decomposition processes in the same samples found controls of N content and litter C:N ratios over decomposition processes \citep{Mooshammer2011, Leitner2011}. [The system is N limited, at least after 6 month.] However, N content is similar in AK and OS, so N content as a single factor can not explain the differences observed.

(2) The same applies for litter DOC content: Higher DOC quantity in SW and AK lead to different trends, in SW lignin was most accumulated in AK the least.

(3) Micro-nutrients are nessesary cofactors for oxidative enzymes and have different contents in the four litter types. Their availability can limit lignin degradation [lit]. However, in AK, their concentration in lower  (Mn, Fe) or equal (Zn) concentrations than in other litter types. Low contents of these Elements would explain inhibited, not enhance lignin decomposition in AK.

We therefore suggest that the ratio between microbially accessible (=dissolved) carbon and litter nitrogen content 

%other lipophilic compounds

\subsection{Changes in decomposition controls over time}

While we found no explaining factor for the initial amount of extractable carbon [beside a loose correlation to litter N content], DOM production is strictly correlated to nitrogen content after six month incubation. Initial DOM amounts show a high independence from other factors [including starch content [check]], DOM production or consumption surpluses increase or decrease the DOM pool during the first 6 month of incubation but then reach an equilibrium point at which DOM content correlates with litter N content.

Nitrogen content is also tightly correlated to respiration beyond 6 month incubation. Unlike proposed by \cite{Klotzbucher2011}, in our experiment respiration was not to be principally controlled by DOC i.e. labile carbon availability, but either both processes are controlled by nitrogen availability or respiration depends on available carbon, which itself is controlled by nitrogen availability as described above. Direct N limitation seems plausible, as de-polymerization of POM compounds depends on extracellular enzymes. Their produce requires large investments of nitrogen from the microbial community. 


Long chain alcanes are among the substance with had the highest increase during the first month of litter decomposition. During the first 3 month their relative peak area increased by 80\%. [Where does these compounds come from?] Fatty acids were the most important inpurity of isolated lignin fractions. They were decomposed faster than lignin, with little differences between litter types [faster in N-poor litter]

%The ratio between glucan depolymerization and respiration shifts between 97 and 181 days. while during the first two harvests, respiration is (relatively) higher in litter types with high DOC and low N content, after 181 days, this ratio is strictly correlated to litter N content, with higher respiration for sites with high N. This ratio allows different interpretations: It might indicate a higher carbon use efficiency on part of microbial communities with a low depolymerization:respiration ratio, or the use of alternative (non-glucose substrates) in litter types with a high ratio.


%H2 - H3. glc depoly (+ aa depoly?) : resp. 

%Another possible explaination of differences between proximate analysis and specific determination of lignin oxidation or pyrolysis products is that first steps of lignin degradation remove characteristic groups from lignin polymers (i.e. methoxy groups), leaving a rest lignin with no specific tracers recognizable with the methods mentioned. This would lead to an underestimation of lignin.

\subsection{Microbial biomass [and decomposition processes]}

After 6 month, AK shows the strongest increase in microbial C. The increase in microbial N is even stronger,  so that after 6 month, AK, a litter type with low N content has the highest microbial N content and the most narrow microbial C:N ratio. This is the time point, when the most lignin is decomposed in AK. We suggest, that this is due to better nitrogen accessibility after increased lignin decomposition. Dissolved C and N pools (organic and inorganic) are one magnitude smaller than microbial biomass pools, and can not harbor de-polymerized litter biomass, which must be (a) respires, (b) incorporated into biomass or (c) immobilized to the POM pool. 

Between 6 and 15 month, lignin does not further accumulate in any site. Microbial metabolisms are adjusted to their substrate, DOC production and consumption are in equilibrium.

Decomposition processes are well correlated to each other and litter N content.We did not, however, find feedback from elevated/depleted lignin content of processes measured.

CN ratios are consistent with the proteomic Fungi/Bacteria ratio. 
\section{Conclusions}

%\conclusions
%% \conclusions[modified heading if necessary]
\cite{Fontaine2011} suggests a ``bank model'' for SOM vs. litter degradation in soils. N-rich recalcitrant carbon is decomposed when N content is low, while N-fertilized soils principally degrade carbohydrate-rich and N-poor litter leachates. This leads to increased N mobilization in N-poor soils when additional (labile) carbon is available. On term of the microbial community, the production of oxidative enzymes is needed to degrade SOM, so investment in the production of these enzymes is up-regulated under C-rich and N-poor conditions. Our results suggest, that similar controls exist in litter decomposition.

%[?Lignin is not rejected for its intrinsic recalcitrance, but because it has little to offer to a community with rare labile carbon..?...]

\section{Acknowledgements}
FWF NRN MicDiF, Katherina Keiblinger DOC fForte
Thanks to Andreas Bl\"ochl, Clemens Schwarzinger, and Birgit Wild for technical advice on Pyr-GC/MS techniques.


%% The Appendices part is started with the command \appendix;
%% appendix sections are then done as normal sections
%% \appendix

%% \section{}
%% \label{}

%% References
%%
%% Following citation commands can be used in the body text:
%%
%%  \citet{key}  ==>>  Jones et al. (1990)
%%  \citep{key}  ==>>  (Jones et al., 1990)
%%
%% Multiple citations as normal:
%% \citep{key1,key2}         ==>> (Jones et al., 1990; Smith, 1989)
%%                            or  (Jones et al., 1990, 1991)
%%                            or  (Jones et al., 1990a,b)
%% \cite{key} is the equivalent of \citet{key} in author-year mode
%%
%% Full author lists may be forced with \citet* or \citep*, e.g.
%%   \citep*{key}            ==>> (Jones, Baker, and Williams, 1990)
%%
%% Optional notes as:
%%   \citep[chap. 2]{key}    ==>> (Jones et al., 1990, chap. 2)
%%   \citep[e.g.,][]{key}    ==>> (e.g., Jones et al., 1990)
%%   \citep[see][pg. 34]{key}==>> (see Jones et al., 1990, pg. 34)
%%  (Note: in standard LaTeX, only one note is allowed, after the ref.
%%   Here, one note is like the standard, two make pre- and post-notes.)
%%
%%   \citealt{key}          ==>> Jones et al. 1990
%%   \citealt*{key}         ==>> Jones, Baker, and Williams 1990
%%   \citealp{key}          ==>> Jones et al., 1990
%%   \citealp*{key}         ==>> Jones, Baker, and Williams, 1990
%%
%% Additional citation possibilities
%%   \citeauthor{key}       ==>> Jones et al.
%%   \citeauthor*{key}      ==>> Jones, Baker, and Williams
%%   \citeyear{key}         ==>> 1990
%%   \citeyearpar{key}      ==>> (1990)
%%   \citetext{priv. comm.} ==>> (priv. comm.)
%%   \citenum{key}          ==>> 11 [non-superscripted]
%% Note: full author lists depends on whether the bib style supports them;
%%       if not, the abbreviated list is printed even when full requested.
%%
%% For names like della Robbia at the start of a sentence, use
%%   \Citet{dRob98}         ==>> Della Robbia (1998)
%%   \Citep{dRob98}         ==>> (Della Robbia, 1998)
%%   \Citeauthor{dRob98}    ==>> Della Robbia


%% References with bibTeX database:

\bibliographystyle{model2-names}
\bibliography{library}

\input{graphs_captions.tex}


\newpage
\begin{table*}[p] 
\begin{tabular}{ccccccc}
Litter type & LCI increase & initial DOC & N & P & Mn & Fe \\
\hline
AK&-&++&-\-&-&-\-&- -\\
KL&++&-\-&-&-\-&+&- -\\
OS&++&-\-&- -&+&-&++\\
SW&++&++&++&++&++&-\-\\
\hline
\end{tabular}
\caption{Summary of context data}
\label{tab:summary}
\end{table*}

%\input{tables.tex}

%\input{graphics.tex}

%% Authors are advised to submit their bibtex database files. They are
%% requested to list a bibtex style file in the manuscript if they do
%% not want to use model2-names.bst.

%% References without bibTeX database:

% \begin{thebibliography}{00}

%% \bibitem must have one of the following forms:
%%   \bibitem[Jones et al.(1990)]{key}...
%%   \bibitem[Jones et al.(1990)Jones, Baker, and Williams]{key}...
%%   \bibitem[Jones et al., 1990]{key}...
%%   \bibitem[\protect\citeauthoryear{Jones, Baker, and Williams}{Jones
%%       et al.}{1990}]{key}...
%%   \bibitem[\protect\citeauthoryear{Jones et al.}{1990}]{key}...
%%   \bibitem[\protect\astroncite{Jones et al.}{1990}]{key}...
%%   \bibitem[\protect\citename{Jones et al., }1990]{key}...
%%   \harvarditem[Jones et al.]{Jones, Baker, and Williams}{1990}{key}...
%%

% \bibitem[ ()]{}

% \end{thebibliography}



\end{document}

%%
%% End of file `elsarticle-template-2-harv.tex'.

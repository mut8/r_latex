\section{Discussion}

\subsection{Intra-specific variance in beech litter and decomposition trends}

We find characteristic patterns of pyrolysis products from different sites. Most important differences were found between furane-type and cyclopentenone-type carbohydrate markers. Also, among the lignin markers, we found differences in the methylguaiacol:guaiacol and methylsyringol:syringol ratio. Differences in the carbohydrate pools possibly origin in different carbohydrates present in litter, while differences in lignin markers maybe indicate different polymerization structures. Alternatively, they can be result of matrix effects during pyrolysis. 

These differences were preserved during litter decomposition, probably due to the low litter decomposition speed observed in beech litter.
%sind eigentlich results

\subsection{Nutrient controls on carbon chemistry}

We found profound differences in patterns of accumulation and depletion of lignin and carbohydrates. During the first 6 month of decomposition, lignin is accumulated and carbohydrates are deplete in three litter types (KL, OS, SW). However, no litter carbon mineralization was not coupled to lignin accumulation or carbohydrate depletion in the forth litter type (AK). Comparing changes in litter chemistry to respiration rates (fig. \ref{fig:lci} (B) and \ref{fig:timeseries} (right side)), we can exclude low litter turnover as a reason for the missing shift in litter chemistry in AK . Especially as OS hat only slightly higher accumulated respiration, but a similar rate of lignin accumulation like SW and KL. This indicates, that - in contrast to the other litter types - there is no microbial substrate preference of carbohydrates over lignin in AK litter and that lignin is decomposed during early litter decay in AK. In other sites, lignin is not decomposed or only at a rate relatively slower than carbohydrates. 

Potential enzyme activities support our findings: N-rich sites had the highest absolute activity for both cellulase and oxidative enzymes. This reflects higher turnover of organic carbon in N-rich litter observed in most decomposition processes [lit maria?]. Unlike some other studies (reviewed by \cite{Sinsabaugh2010} - [check if fertilization experiments]) we did not find an inhibition of oxidative enzymes in absolute terms under high (natural) N content in the substrate. The absolute amount of enzymes produced [might be] limited by N availability and is strongly correlated with other decomposition processes [provide stats]. Unlike the absolute amount of enzymes produced, the ratio between cellulose hydrolyzing and oxidative enzymes is lower in AK than in other sites. Investments of the microbial community are directed more into degrading lignin in AK than in other sites.

%Unlike cellulose and protein, degradation of lignin does not yield a single specific monomer. Due to this unspecific biochemistry, it is not possible to specifically measure lignin decomposition speed by a pool dilution method. Nevertheless, the ratio between glucose depolymerization and respiration allows an estimation, to which extent non-glucose carbon is respired by litter microbes. 

%Several independend methods show similar indication: analytical pyrolysis, calculation of non-glucose respiration, potential enzyme activities. 

The early lignin decomposition concept recently presented by \cite{Klotzbucher2011} seems fit for one litter type (AK), but not for the other three. Several possible reasons for stimulated/inhibited lignin decomposition were suggested in recent literature:

(1) Litter nitrogen content was strictly correlated to most decomposition processes measured [enzymes, N-depoly, Glucose-depoly, ... ] after 6 month and [test!]correlated to respiration at earlier harvest. Earlier analysis of decomposition processes in the same samples found controls of N content and litter C:N ratios over decomposition processes \citep{Mooshammer2011, Leitner2011}. [The system is N limited, at least after 6 month.] However, N content is similar in AK and OS, so N content as a single factor can not explain the differences observed.

(2) The same applies for litter DOC content: Higher DOC quantity in SW and AK lead to different trends, in SW lignin was most accumulated in AK the least.

(3) Micro-nutrients are nessesary cofactors for oxidative enzymes and have different contents in the four litter types. Their availability can limit lignin degradation [lit]. However, in AK, their concentration in lower  (Mn, Fe) or equal (Zn) concentrations than in other litter types. Low contents of these Elements would explain inhibited, not enhance lignin decomposition in AK.

We therefore suggest that the ratio between microbially accessible (=dissolved) carbon and litter nitrogen content 

%other lipophilic compounds

\subsection{Changes in decomposition controls over time}

While we found no explaining factor for the initial amount of extractable carbon [beside a loose correlation to litter N content], DOM production is strictly correlated to nitrogen content after six month incubation. Initial DOM amounts show a high independence from other factors [including starch content [check]], DOM production or consumption surpluses increase or decrease the DOM pool during the first 6 month of incubation but then reach an equilibrium point at which DOM content correlates with litter N content.

Nitrogen content is also tightly correlated to respiration beyond 6 month incubation. Unlike proposed by \cite{Klotzbucher2011}, in our experiment respiration was not to be principally controlled by DOC i.e. labile carbon availability, but either both processes are controlled by nitrogen availability or respiration depends on available carbon, which itself is controlled by nitrogen availability as described above. Direct N limitation seems plausible, as de-polymerization of POM compounds depends on extracellular enzymes. Their produce requires large investments of nitrogen from the microbial community. 


Long chain alcanes are among the substance with had the highest increase during the first month of litter decomposition. During the first 3 month their relative peak area increased by 80\%. [Where does these compounds come from?] Fatty acids were the most important inpurity of isolated lignin fractions. They were decomposed faster than lignin, with little differences between litter types [faster in N-poor litter]

%The ratio between glucan depolymerization and respiration shifts between 97 and 181 days. while during the first two harvests, respiration is (relatively) higher in litter types with high DOC and low N content, after 181 days, this ratio is strictly correlated to litter N content, with higher respiration for sites with high N. This ratio allows different interpretations: It might indicate a higher carbon use efficiency on part of microbial communities with a low depolymerization:respiration ratio, or the use of alternative (non-glucose substrates) in litter types with a high ratio.


%H2 - H3. glc depoly (+ aa depoly?) : resp. 

%Another possible explaination of differences between proximate analysis and specific determination of lignin oxidation or pyrolysis products is that first steps of lignin degradation remove characteristic groups from lignin polymers (i.e. methoxy groups), leaving a rest lignin with no specific tracers recognizable with the methods mentioned. This would lead to an underestimation of lignin.

\subsection{Microbial biomass [and decomposition processes]}

After 6 month, AK shows the strongest increase in microbial C. The increase in microbial N is even stronger,  so that after 6 month, AK, a litter type with low N content has the highest microbial N content and the most narrow microbial C:N ratio. This is the time point, when the most lignin is decomposed in AK. We suggest, that this is due to better nitrogen accessibility after increased lignin decomposition. Dissolved C and N pools (organic and inorganic) are one magnitude smaller than microbial biomass pools, and can not harbor de-polymerized litter biomass, which must be (a) respires, (b) incorporated into biomass or (c) immobilized to the POM pool. 

Between 6 and 15 month, lignin does not further accumulate in any site. Microbial metabolisms are adjusted to their substrate, DOC production and consumption are in equilibrium.

Decomposition processes are well correlated to each other and litter N content.We did not, however, find feedback from elevated/depleted lignin content of processes measured.

CN ratios are consistent with the proteomic Fungi/Bacteria ratio. 